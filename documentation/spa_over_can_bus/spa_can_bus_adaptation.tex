\section{SPA CAN Bus Adaptation}
Each problem from chapter \ref{ch:problem_statement} is addressed in this
chapter with corresponding subsection headlines.

\subsection{Bootstraping}
According to SPA each subnet should have its own subnet manager, for the CAN Bus
subnet that component is called SM-c (SPA Manager CAN Bus). When the SM-c has
been assigned an address block from the CAS component the SM-c must assign a
logical address to each component on its subnet. Each logical address must also
be mapped to a local address specific for the CAN Bus subnet.

A straightforward solution is to assign one CAN Bus Message ID for each
component that it should listen on. The problem with this is that if more than
one component tries to contact another component, at the same time, both will
start their respective message with the same CAN Message ID. This will cause
a bit error on one of the components that in may cause it switch to "bus off" state. To
reactivate the components that have gone into "bus off" state, special messages
must be sent over the CAN Bus. To make sure that this doesn't happen each pair
of components should be assigned a unique pair of CAN Bus Message IDs that they
can communicate over.

% Number of Message IDs required for this?

With the mapping of SPA logical addresses to local CAN Bus subnet addresses out
of the way, it's time to look at the actual assignment. During "Network Topology
Discovery" within a SPA-Local interconnect all components can contact the SM-l
component through a Well-known port of 3500 \cite{standard:spa_local_adaptation}.
Applying a similiar approach on a CAN Bus where all connected components contact
the SM-c with a well-known CAN Bus Message ID would cause similiar problems as
described earlier.

% Continue with a suitable example from SpaceWire Adaptation Standard
